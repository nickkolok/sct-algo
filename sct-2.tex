\documentclass[a4paper,14pt]{article}
\usepackage[T2A]{fontenc}
\usepackage[utf8]{inputenc} % любая желаемая кодировка
\usepackage[russian,english]{babel}
\usepackage[pdftex,unicode]{hyperref}
\usepackage{indentfirst} % включить отступ у первого абзаца
\usepackage{amssymb}
\usepackage{amsthm}
\usepackage{amsmath}

\newcommand{\N}{\mathbb{N}}
\newcommand{\Z}{\mathbb{Z}}
\newcommand{\Q}{\mathbb{Q}}
\newcommand{\R}{\mathbb{R}}
\begin{document} % начало документа
\large

Определение.

Множество точек $M$ называется системой целочисленно удалённых точек (СЦТ), если
$$
\forall(M_1 \in M, M_2\in M)[|M_1 M_2|\in\Z],
$$
и при этом $M$ не содержится целиком ни в какой прямой.

Замечание.

Мы рассматриваем случай точек на плоскости, т. е. в $\R^2$.


Замечание.

Прямую, разбитую точками на целочисленные отрезки, мы, как видно из определения, не рассматриваем.
Аналогично не представляют для нас интереса множества, состоящие из одной или двух точек.

Определение.

Количичество точек в СЦТ $S$ называется её мощностью $P(S)$.

Лемма 1 (доказана в теме <<Применение фокального свойства гиперболы>> \cite{angem1kurs})

Если в СЦТ $S$ три точки $M_1$, $M_2$ и $M_3$ не лежат на одной прямой и 
$a=|M_1 M_2| \in \mathbb{N}$,
$b=|M_1 M_3| \in \mathbb{N}$,
$c=|M_2 M_3| \in \mathbb{N}$,
то 
$P(S) \leq 4\cdot\min\{ab,ac,bc\}$.

Следствие.

Любая СЦТ конечна.

Лемма 2.

Если в СЦТ $S$ найдётся $\beta = 2m^2 +1$ точек, никакие три из которых не лежат на одной прямой, и $S$ лежит в пределах квадрата со стороной $n$, то $n > \frac{\beta - 1}{4}$ (иначе говоря, $ \beta < 4n +1$).

Доказательство

СЦТ лежит в пределах квадрата со стороной $n$. Разобьём этот квадрат на $m^2$ меньших равных между собой квадратов со стороной $\frac{n}{m}$. Тогда по принципу Дирихле найдётся хотя бы один квадрат со стороной $\frac{n}{m}$, внутри которого (возможно, включая границы) найдутся три точки, принадлежащие рассматриваемой СЦТ. Обозначим их через $M_1$, $M_2$ и $M_3$. Ни одно из расстояний $|M_1 M_2|$, $|M_1 M_3|$ и $|M_2 M_3|$, очевидно, не превышает диагонали квадрата со стороной $\frac{n}{m}$, т. е. $\frac{n}{m}\sqrt{2}$. Тогда по лемме 1 количество точек в СЦТ $\beta \le \frac{8n^2}{m^2}$. Имеем:
$$ 2m^2+1 \le \frac{8n^2}{m^2}$$
$$ 2m^2 < 2m^2+1 \le \frac{8n^2}{m^2}$$
$$ 2m^2 < \frac{8n^2}{m^2}$$
$$ m^2 < \frac{4n^2}{m^2}$$
$$ m^4 < 4n^2$$
Т. к. $n$ положительно, извлекаем корень:
$$ m^2 < 2n$$
$$ 2m^2 +1 < 4n +1$$
$$ \beta < 4n +1$$
$$n > \frac{\beta - 1}{4}$$
Лемма доказана.

Утверждение 1 (вспомогательное)

$\forall \left(\beta \in \mathbb N\right)\left[  2 \sqrt{\beta - 1} \leq \beta \right]$

Доказательство

Т. к. $\beta >0$, возводим обе части неравенства в квадрат:

$$4 (\beta - 1) \leq \beta^2$$
$$ (\beta-2)^2 \geq 0$$

Утверждение доказано.

Определение.

Назовём СЦТ $S$ бестриадной, если никакие три точки из $S$ не лежат на одной прямой.


Определение.

Большим диаметром $D(S)$ СЦТ $S$ называется диаметр наименьшего круга, целиком покрывающего СЦТ $S$.

Определение.

Малым диаметром $d(S)$ СЦТ $S$ называется максимум из попарных расстояний между её точками.

Заметим, что оба диаметра, во-первых, определены корректно, во-вторых, конечны, в третьих, связаны соотношением $d(S)\leq D(s) \leq 2d(s)$ (последнее неравенство вытекает из того, что, выбрав точку $A$ из произвольной пары максимально удалённых точек, можно построить круг радиуса $d(S)$ с центром в точке $A$, который, очевидно, покроет всю СЦТ).


Лемма 3.

Пусть СЦТ $S$ --- бестриадна и $S$ лежит внутри квадрата со стороной $n$, $P(S)=\gamma$.
Тогда $\gamma<4(1+\sqrt{2})n+2+\sqrt{2}$.

Доказательство.

Возьмём $m \in \mathbb{N}$ такое, что $2m^2+1 \le \gamma \le 2(m+1)^2$ (это можно сделать единственным образом).
Обозначим $2m^2+1=\beta$, откуда $m=\sqrt{\frac{\beta-1}{2}}$. Тогда по лемме 2 имеем $ \beta < 4n +1$. Оценим $\gamma$:

\begin{multline}
\gamma \le 2(m+1)^2 = 2m^2+4m +2 \leq
\beta + 1 + 2 \cdot 2 \sqrt{\frac{\beta-1}{2}} =
\beta + 1 +2\sqrt{2}\sqrt{\beta-1} \leq
\\ \leq
(1+\sqrt{2})\beta+1 <
(4n+1)(1+\sqrt{2})+1 =
4(1+\sqrt{2})n+2+\sqrt{2}
\end{multline}

Лемма доказана.

Следствие.

Если СЦТ $S$ бестриадна, то
$P(S) < 10 D(S)+4$.
Заметим, что это очень грубое ограничение, указывающее, однако, на не более чем линейный характер зависимости максимальной возможной мощности  бестриадной СЦТ от её диаметра.

В \cite{angem1kurs} Е.М. Семёнов даёт способ построения СЦТ с произвольной наперёд заданной мощности, не приводя, однако, зависимость диаметра СЦТ, получаемой при таком построении, от её мощности.
Приведём здесь оригинальный способ построения СЦТ, конструктивно доказав следующую лемму:

Лемма 4.

Для любого натурального $k$ найдётся СЦТ $S$ такая, что $P(S)=2k+3$, $D(S)\leq 2^{4k+1}$.

Доказательство.

Известно, что $(2mn)^2+(m^2-n^2)^2=(m^2+n^2)^2$.
Заметим, что $2^{2k+1} = 2 \cdot 2^{2k-p} \cdot 2^p$, где $p$ --- целое число от $0$ до $k-1$.
Таким образом, мы получили представление числа $2^{2k+1}$ в виде $2mn$ $k$ способами.
Рассмотрим теперь множество точек $S=\{O=(0;0),B_\pm=(0;\pm 2^{2k+1}),A_{\pm p}(\pm 2^{2(2k-p)}-2^{2p};0)\}$.
Покажем, что $S$ - СЦТ.
Понятно, что $|O-B_\pm|\in\Z$, $|O-A_{\pm p}|\in\Z$.
Убедимся, что $|B_\pm-A_{\pm p}|\in\Z$.
Действительно, $|B_\pm-A_{\pm p}|=|(0;\pm 2^{2k+1}) - (\pm 2^{2(2k-p)}-2^{2p};0)| = \sqrt{(2^{2k+1})^2+(2^{2(2k-p)}-2^{2p})^2}=
\sqrt{2^{4k+2}+2^{4(2k-p)}-2\cdot 2^{4k}+2^4p}=\sqrt{2^{4(2k-p)}+2\cdot 2^{4k}+2^4p}=2^{2(2k-p)}+2^{2p}\in\Z$
Мощность СЦТ $S$ равна в точности $2k+3$.
Построим круг с центром в $O$ радиуса $2^{4k}$.
Заметим, что $|O-B_\pm|=2^{2k+1}<2^{4k}$, $|O-A_{\pm p}| = 2^{2(2k-p)}-2^{2p} < 2^{4k}$.
Значит, построенный круг диаметра $2^{4k+1}$ покрывает СЦТ $S$.

Лемма доказана.


\addcontentsline{toc}{chapter}{Литература}
\begin{thebibliography}{99}

\bibitem{angem1kurs} Аналитическая геометрия на плоскости / Е.М. Семенов, С.Н. Уксусов. – Воро-неж : Воронежский государственный университет, 2013. – 100с.

\end{thebibliography}
\end{document} % конец документа
